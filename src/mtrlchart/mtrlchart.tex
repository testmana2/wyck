%\VignetteIndexEntry{mtrlchart: market analytics using Wyckoff methodology}
\documentclass{article}
\usepackage{hyperref}
\usepackage[page]{appendix}
\hypersetup{colorlinks,%
            citecolor=black,%
            linkcolor=blue,%
            urlcolor=blue,%
            }

\title{\bf mtrlchart: Market analytics with Wyckoff meethodology }
\author{Sviatoslav Zimine}
\date{May 27, 2015}

\usepackage{Sweave}
\begin{document}

\maketitle
\tableofcontents
%%%%%%%%%%%%%%%%%%%%%%%%%%%%%%%%%%%%%%%%%%%%%%%%%%%%%%%%%%%%%%%%%%%%%%%%%%%%%%%%%%
%%%%%%%%%%%%%%%%%%%%%%%%%%%%%%%%%%%%%%%%%%%%%%%%%%%%%%%%%%%%%%%%%%%%%%%%%%%%%%%%%%
\section{Introduction}
The package {\tt mtrlchart} was created to perform analysis on market time-series based on combinations of price and volume action and following a methodology originally developed by {\sl Richard Wyckoff}~\cite{richardwyckoff} in the beginning of the 20th century. The training based on original methodology is provided by the Wyckoff Stock Market Institute~\cite{wyckoffinst}.

A cousin package {\tt pmktdata} was developed to support dowloading of market data frm {\sl Yahoo} and {\sl DTN} providers.

A user loads from an {\sl .rdat} file and manipulates a {\tt Wwatchlist} object which contains a list of instruments with downloaded market data. 
%%%%%%%%%%%%%%%%%%%%%%%%%%%%%%%%%%%%%%%%%%%%%%%%%%%%%%%%%%%%%%%%%%%%%%%%%%%%%%%%%%
%%%%%%%%%%%%%%%%%%%%%%%%%%%%%%%%%%%%%%%%%%%%%%%%%%%%%%%%%%%%%%%%%%%%%%%%%%%%%%%%%%
\section{Working with an instrument watchlist}

\subsection{Loading a watchlist}
A user first loads  a watchlist previously  prepared and stored in {\sl .rdat} file in a directory on a file system (note: in my Rsession loading the mtrchart package and its dependencies requires just {\sl require(mtrlchart)})

\begin{Schunk}
\begin{Sinput}
> ##loading this package
> require(mtrlchart, lib.loc='~/R/library')
> ## loading  the watchlist rdat file
> wl <- lwl('~/googledrive/cdat/wls/geaa_rnw.rdat')
\end{Sinput}
\begin{Soutput}
loading mdata from separate file  /Users/zimine/googledrive/cdat/wls/geaa_rnw_mldata.rdat 
loading watchlist from file  ~/googledrive/cdat/wls/geaa_rnw.rdat 
\end{Soutput}
\end{Schunk}
The package contains a number of helper functions grouped by subjects.

\begin{Schunk}
\begin{Sinput}
> ohelp(wl)
\end{Sinput}
\begin{Soutput}
#############################################################
** Wyckof artifacats **
#############################################################
whelp(o) ::	  help on Wyckoff registered set of artifacts
#############################################################
*** Market data manipulation functions  ***
#############################################################
wldhelp(o) ::	  help on attaching, forwarding market data 
#############################################################
***Artifact manipulation functions***
#############################################################
wfhelp(o) ::	  help on Wyckoff artifacts manipulation
#############################################################
**Charting manipulation functions**
#############################################################
crthelp(o) ::	 help on charting wyckoff facts
#############################################################
**Report generation functions**
#############################################################
prephelp(o) ::	 help on generating print report
#############################################################
**Trade Order management functions**
#############################################################
trhelp(o) ::	  help on trade , order management user functions

#############################################################
**Help on help **
#############################################################
ohelp(o, flat=TRUE) ::	  print all help files 
#############################################################
\end{Soutput}
\end{Schunk}
To see the contents of the loaded instruments in a watchlist and their respective downloaded market data print the object:

\begin{Schunk}
\begin{Sinput}
> wl
\end{Sinput}
\end{Schunk}
\subsubsection{Switching between two modes}
a {\tt Wwatchlist} object supports two modes. A {\sl live} mode is suitable for up to date data downloads and curent daily analysis. A {\sl historic} mode is useful for data analysis and walk-through in the past for training purposes.

\begin{Schunk}
\begin{Sinput}
> ##switch to live mode
> dmode(wl, mode='live', startdate='1980-01-01', enddate=Sys.date())
> ##print current instrument last data
> pd(wl,n=1)
\end{Sinput}
\begin{Soutput}
ohlc tail data for GE timeframe= d1 
           d1_xohlc.Open d1_xohlc.High d1_xohlc.Low d1_xohlc.Close d1_xohlc.Volume d1_xohlc.Adjusted
2014-11-10         26.41         26.53        26.38          26.47        16253600             26.47
\end{Soutput}
\begin{Sinput}
> ##switch to historic mode
> dmode(wl, mode='historic', startdate='1980-01-01', enddate=Sys.Date() )
> ##go to specific date in the past
> dcdate(wl,date='1985-01-01')
\end{Sinput}
\begin{Soutput}
[1] "current date: 1985-01-02 Wednesday"
\end{Soutput}
\begin{Sinput}
> pd(wl)
\end{Sinput}
\begin{Soutput}
ohlc tail data for GE timeframe= d1 
           d1_xohlc.Open d1_xohlc.High d1_xohlc.Low d1_xohlc.Close d1_xohlc.Volume d1_xohlc.Adjusted
1984-12-24      1.009940      1.023375     1.009940       1.023375         9158400              1.02
1984-12-26      1.025524      1.025524     1.012090       1.016568         4036800              1.02
1984-12-27      1.016568      1.021046     1.009940       1.012090         4761600              1.01
1984-12-28      1.014418      1.016568     1.009940       1.012090         5937600              1.01
1984-12-31      1.016568      1.018717     1.012090       1.014418         5313600              1.01
1985-01-02      1.009940      1.012090     1.000804       1.003133         5925600              1.00
\end{Soutput}
\end{Schunk}

\subsubsection{Navigating market data}
Forwarding market data applies mostly to the {\sl historic mode}. After setting a current date with {\tt dcdate(wl)}, use {\tt gnd(wl)}, {\tt gnw(wl)} to advance by 1 day, 1 week.
\begin{Schunk}
\begin{Sinput}
> gnd(wl)
\end{Sinput}
\begin{Soutput}
[1] "current date: 1985-01-03 Thursday"
\end{Soutput}
\begin{Sinput}
> pd(wl,n=1)
\end{Sinput}
\begin{Soutput}
ohlc tail data for GE timeframe= d1 
           d1_xohlc.Open d1_xohlc.High d1_xohlc.Low d1_xohlc.Close d1_xohlc.Volume d1_xohlc.Adjusted
1985-01-03      1.000804      1.018717    0.9965052       1.007611        13716000              1.01
\end{Soutput}
\end{Schunk}
Use {\tt gni(wl),gif(wl), gis(wl,instr)} to switch a current instrument to next, first or a specific instrument in the watchlist.

\subsection{configure a watchlist}
An instrument watch list has a csv configuration file \texttt{ge\_aa\_ref.csv}. Further to an instrument type, this configuration contains a classification for the equity stocks grouping.
\begin{Schunk}
\begin{Soutput}
  instr ticker type            label      group X
1  GSPC  ^GSPC   ix    S&P 500 index         ix 1
2    GE     GE   eq General Electric industrial 1
3    AA     AA   eq        Alcoa Inc     metals 1
\end{Soutput}
\end{Schunk}
This configuration is used to create a list of instrument and download  the corresponding market data
using a function {\tt instrlistdatacsv()}. Full sequence of commands  is in a script \texttt{mk\_geaa\_ref.R} presented in the appendix.

\section{Working with labels and levels}
A list of Wyckoff labels is available from {\tt whelp(wl)} \cite{wyckoffinst}. Refer to appendix

\subsection{Charting}

\begin{center}
Refer to {\tt crthelp(wl)} function to list charting functions. To chart current instrument on a daily frame use {\tt ccd()}, for weekly timeframe {\tt ccw()}.
\begin{Schunk}
\begin{Sinput}
> ccd(wl)
\end{Sinput}
\begin{Soutput}
[1] "No artefacts defined for current instrument."
CURRENT DATE:  1985-01-03 UTC Thursday 
\end{Soutput}
\end{Schunk}
\includegraphics{mtrlchart-ccd}
\end{center}

% To chart instruments defined as your active portoflio
% \SweaveOpts{height=12,width=12}
% \begin{center}
% <<ccm, fig=TRUE>>=
% ccm(wl)
% @
% \end{center}

%%%%%%%%%%%%%%%%%%%%%%%%%%%%%%%%%%%%%%%%%%%%%%%%%%%%%%%%%%%%%%%%%%%%%%%%%%%%%%%%%%
%%%%%%%%%%%%%%%%%%%%%%%%%%%%%%%%%%%%%%%%%%%%%%%%%%%%%%%%%%%%%%%%%%%%%%%%%%%%%%%%%%
\section{Trading simulator}
\subsection{Configure the active portfolio}
First, edit a tradebook configuration csv file. Specify instrument ticket, currency, tick size, tick dollar value, tick unit brokerage fee in dollars, initial margin pert contract in dollars, slippage in ticks, relative weight of a security in a portfolio (1,1,1 are  three equally weighted securities).
In our example it is called \texttt{geaa\_tradebook\_cfg\_ref.csv}.
\begin{Schunk}
\begin{Soutput}
  instr type currency tksize tkval  ufee uimargin slipp rweight
1  GSPC   ix      USD   0.25 12.50 2.150        1   1.0     0.0
2    AA   eq      USD   0.01  0.01 0.002        1   0.5     1.2
3    GE   eq      USD   0.01  0.01 0.002        1   0.5     0.8
\end{Soutput}
\end{Schunk}
Full list of commands below is saved in a script \texttt{add\_geaa\_tradebook.R} presented in the appendix.
\subsubsection{Common operations}
A first operation usually involves checking an attractiveness of a potential trade for risk/reward ratio. Check it using {\tt tdck(wl,ft=100.00, mt=110.00)}. This function requires to have a chart with an instrument used to specify interactively an entry price level and stop level. {\tt ft, mt} are parameters for first target and main target of a potential trade.
\begin{Schunk}
\begin{Soutput}
[1] "No artefacts defined for current instrument."
CURRENT DATE:  1985-01-03 UTC Thursday 
\end{Soutput}
\end{Schunk}
\begin{Schunk}
\begin{Sinput}
> tdck(wl,pent=2.41, pstop=2.31, ft=2.45, mt=2.90)
\end{Sinput}
\end{Schunk}
The results of this check are stored in a temporary trade which one can print with {\tt tdpt()}.
\begin{Schunk}
\begin{Sinput}
> tdpt(wl) ## print details of the potential trade, ftrr, mtrr are risk reward ratios
\end{Sinput}
\begin{Soutput}
temporary trade details:
       strat acc tkr    incdate status   qty entry stop ftarg mtarg
tid: 0     1  10  GE 1985-01-03      n 10000  2.41 2.31  2.45   2.9
.....................................................................
       ftrr mtrr pos stopSet TargetSet Upnl Rpnl orphaned
props:  0.4  4.9   0      no        no    0    0       no
comment: NA 
.....................................................................
\end{Soutput}
\end{Schunk}
To place an order for a current instrument (market, or limit) use 
\begin{Schunk}
\begin{Sinput}
> tdpl(wl,qty=1000,mkto=FALSE,plstop=TRUE)
\end{Sinput}
\begin{Soutput}
[1] "trade opened with limit working order"
 ** Order **  id: 0 trid: 0 instr: GE type: l status: w chdate: 1985-01-03 qty: 1000 rprice: 2.4100 fprice: 0.0000 
\end{Soutput}
\end{Schunk}
with order size suggested by {\tt tdck()} determined by the risk per trade {\sl risktradepct} parameter
from the order book configuration.\ref{trap} 
To check if a placed order is filled on next bar use {\tt gnd(wl); cfill(wl)} or  an equivalent {\tt gnf(wl)}.
\begin{Schunk}
\begin{Sinput}
> gnf(wl)
\end{Sinput}
\begin{Soutput}
[1] "current date: 1985-01-04 Friday"
\end{Soutput}
\end{Schunk}
To print the current trades list use
\begin{Schunk}
\begin{Sinput}
> tdp(wl)
\end{Sinput}
\end{Schunk}
you can cancel the trade if it has no exposure and only working limit orders
\begin{Schunk}
\begin{Sinput}
> tdca(wl,tid=1) # tid is a numeric id for each trade
\end{Sinput}
\end{Schunk}
%%%%%%%%%%%%%%%%%%%%%%%%%%%%%%%%%%%%%%%%%%%%%%%%%%%%%%%%%%%%%%%%%%%%%%%%%%%%%%%%%%
%%%%%%%%%%%%%%%%%%%%%%%%%%%%%%%%%%%%%%%%%%%%%%%%%%%%%%%%%%%%%%%%%%%%%%%%%%%%%%%%%%
%%%%%%%%%%%%%%%%%%%%%%%%%%%%%%%%%%%%%%%%%%%%%%%%%%%%%%%%%%%%%%%%%%%%%%%%%%%%%%%%%%
%%%%%%%%%%%%%%%%%%%%%%%%%%%%%%%%%%%%%%%%%%%%%%%%%%%%%%%%%%%%%%%%%%%%%%%%%%%%%%%%%%

\begin{appendices}

\section{Working with an instrument watchlist}
Contents of the script \texttt{mk\_geaa\_ref.R} to create a watchlist from scratch
\begin{Schunk}
\begin{Sinput}
> library(quantmod)
> library(mtrlchart)
> grpsorted <- c("ix", "energy","metals","consumer" ,"financial"
+                       ,"healthcare", "industrial", "materials"
+                       , "technology", "utilities")
> sdt <- "1980-01-01"
> edt <- "2015-10-02"
> csvfile <- "~/Dropbox/cs/osx/osxwr/mtrlchart-batch/wyck-cre/ge_aa_ref.csv"
> #create and download market data for the list from the csvfile
> mktd <- instrlistdatacsv(csvfile, grpsorted,  mode='live'
+                         ,startdate=sdt,enddate=edt
+                         , dsrc='yahoo', dld=T)
> #Create Watchlist object
> wl <- wwlistfromdata(mktd)
> setRdatfile(wl) <- '~/googledrive/cdat/wls/geaa-ref.rdat'
> #persistence
> persistMdata(wl)
> persist(wl,withmarketdata=FALSE,verbose=TRUE)
\end{Sinput}
\end{Schunk}

\section{Trading Simulator}
\label{trap}
Contents of the script \texttt{add\_geaa\_tradebook.R}

\begin{Schunk}
\begin{Sinput}
> #orderbook config file
> cfgtradefname <- '~/Dropbox/cs/osx/osxwr/mtrlchart-batch/order/geaa_tradebook_cfg_ref.csv'
> #values metadata
> portf <- "eqportf"
> startcapital <- 100000  # initial trading capital nav
> iniDate <- "1990-01-01"  #earliest trading date
> risktr <- 0.01  # risk per trade as a percent of total NAV
> acc <- 10
> strat <- 1    
> activeinstrs <- c("AA","GE") # set traded  instruments
> ppars <- initPorfolioParams(accid=acc, stratid=strat, pname=portf, iniDate=iniDate
+                                 ,iniEquity=startcapital, risktradepct=risktr
+                                 ,actinstrs=activeinstrs )
> ob <- initOrderbookcsv(ppars,cfgtradefname)
> setWatchInstruments(ob)<-c("AA","GE")
> wl@orderbook <- ob
> persist(wl,withmarketdata=FALSE,verbose=TRUE)
\end{Sinput}
\end{Schunk}

\section{Working with labels and levels}
The list of Wyckoff facts \cite{wyckoffinst}
\begin{Schunk}
\begin{Sinput}
> whelp(wl)
\end{Sinput}
\begin{Soutput}
***Class Wdict. Dictionary of Wyckoff labels:
**List of Label types:
      [,1]  [,2]                 
 [1,] "WYF" "Wyckoff fact"       
 [2,] "TRL" "Trend line"         
 [3,] "LVL" "Creek Level line"   
 [4,] "ICL" "ICE Level line"     
 [5,] "HFL" "Half Level line"    
 [6,] "BVW" "Bar View fact"      
 [7,] "BPR" "Bar properties fact"
 [8,] "HYP" "Hypo fact"          
 [9,] "NOB" "Note fact"          
**List of Wyckoff labels:
      [,1]   [,2]                                [,3]
 [1,] "SOS"  "Sign of Strength"                  "1" 
 [2,] "SOW"  "Sign of Weakness"                  "-1"
 [3,] "SC"   "Selling Climax"                    "1" 
 [4,] "TSC"  "Secondary Test of Selling Climax"  "1" 
 [5,] "BC"   "Buying Climax"                     "-1"
 [6,] "TBC"  "Secondary Test of Bying Climax"    "-1"
 [7,] "ARN"  "Automatic Reaction"                "1" 
 [8,] "AR"   "Automatic Rally"                   "-1"
 [9,] "NR"   "Normal Reaction"                   "1" 
[10,] "SR"   "Sluggish Rally"                    "-1"
[11,] "SA"   "Stopping action"                   "1" 
[12,] "TSA"  "Secondary test of Stopping action" "1" 
[13,] "UT"   "Upthrust"                          "-1"
[14,] "TUT"  "Secondary test of Upthrust"        "-1"
[15,] "SOT"  "Shortening of Thrust"              "-1"
[16,] "SHK"  "Shake Out"                         "1" 
[17,] "PS"   "Preliminary Support"               "1" 
[18,] "PSY"  "Preliminary Supply"                "-1"
[19,] "LPS"  "Last point of Support"             "-1"
[20,] "LPSY" "Last point of Supply"              "1" 
[21,] "JOC"  "Jump Over Creek"                   "1" 
[22,] "BTC"  "Back to Creek"                     "1" 
[23,] "TBTC" "Secondary test of Back to Creek"   "1" 
[24,] "BUI"  "Break Under Ice"                   "-1"
[25,] "BTI"  "Back to Ice"                       "-1"
[26,] "TBTI" "Secondary test of Back to Ice"     "-1"
[27,] "SPR"  "Spring"                            "1" 
[28,] "TSPR" "Secondary Test of Spring"          "1" 
**List of Feautures contract months:
 [1] "F" "G" "H" "J" "K" "M" "N" "Q" "U" "V" "X" "Z"
\end{Soutput}
\end{Schunk}

\end{appendices}

\begin{thebibliography}{99}

\bibitem{wyckoffinst} Wyckoff stock market institute:
 http://wyckoffstockmarketinstitute.com

\bibitem{richardwyckoff} Richard Wyckoff:
 URL http://en.wikipedia.org/wiki/Richard\_Wyckoff


\end{thebibliography}
\end{document}
